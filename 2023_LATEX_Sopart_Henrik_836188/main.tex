\documentclass[a4paper, 11pt]{article} % Documentklasse und Papierformat festlegen

\usepackage[left=2.5cm, right=2.5cm]{geometry} % Ränder festlegen
\usepackage{fancyhdr} % Um Kopf- und Fußzeilen zu gestalten
\usepackage[utf8]{inputenc} % Encoding für Umlaute
\usepackage[T1]{fontenc} % Font Encoding für korrekte Darstellung von Sonderzeichen
\usepackage[ngerman]{babel} % Sprachpaket für deutsche Sprache
\usepackage{lmodern} % Schriftart
\usepackage{amsmath} % Mathematische Symbole und Funktionen
\usepackage{amssymb} % Mathematische Symbole und Funktionen
\usepackage{graphicx} % Einbinden von Grafiken
\usepackage{caption} % Formatierung von Abbildungs- und Tabellenbeschriftungen anpassen
\usepackage{booktabs} % Erstellen von "schöneren" Tabellen
\usepackage{xcolor} % Farben

\setlength{\parindent}{0em} % Verhindert das Einrücken
\graphicspath{{Img/}} % Pfad zu Grafikken für den Compiler
\captionsetup[figure]{justification=raggedright,singlelinecheck=false} % "justification=raggedright" -> Beschriftungen linksbündig ; singlelinecheck=false -> keine automatische Zentrierung

% Neue Befehlsdefinitionen, um das Erstellen des Deckblattes zu erleichtern und einfacher Anpassungen vornehmen zu können.
\newcommand{\Thema}{FPV Drohnen}
\newcommand{\Unterthema}{Herausforderungen für Hard- und Software}
\newcommand{\Autor}{Henrik Sopart}
\newcommand{\Datum}{\today}




\begin{document}

\begin{titlepage}
    \vspace*{2cm}
    \begin{center}
    {\Huge\bfseries\Thema} \\
    \vspace*{0.8cm}
    {\huge\Unterthema}
    \end{center}
    \vspace*{4cm}

    \begin{table}[h]
        \renewcommand{\arraystretch}{2}
        \begin{tabular}{p{6cm}p{10cm}}
            {\large\textbf{Vorgelegt durch}} & {\large\Autor} \\
            {\large\textbf{Matrikelnummer}} & {\large\Matrikelnummer} \\
            {\large\textbf{Studiengang}} & {\large\Studiengang} \\
            {\large\textbf{Wohnort}} & {\large\Wohnort} \\
            \vspace*{0.4cm} & \vspace*{0.4cm} \\
            {\large\textbf{Art der Arbeit}} & {\large\ArtDerArbeit} \\
            \vspace*{0.4cm} & \vspace*{0.4cm} \\
            {\large\textbf{Abgabedatum}} & {\large\Abgabedatum} \\
            {\large\textbf{Bearbeitungszeitraum}} & {\large\Bearbeitungszeitraum} \\
            {\large\textbf{Prüfer}} & {\large\Pruefer} \\
        \end{tabular}
        \label{tabelle_deckblatt}
    \end{table}




    \end{titlepage}

\tableofcontents
\thispagestyle{empty}
\setcounter{page}{0}


%%% Henrik Sopart, 836188, Latex-Projekt %%%%

\section[Einleitung]{Einleitung}
    Drohnen, auch bekannt als unbemannte Luftfahrzeuge oder UAVs (Unmanned Aerial Vehicles),
    haben in den letzten Jahren enorm an Popularität gewonnen und Perspektiven erlaubt, welche
    vorher unmöglich schienen. Sie werden in der Landwirtschaft eingesetzt, um gezielt Düngemittel
    zu verteilen oder Schädlinge zu bekämpfen. Auch in der Vermessungstechnik haben sich Drohnen
    bewährt, um große Flächen schnell und präzise zu vermessen. In der Unterhaltungsindustrie werden
    sie oft für spektakuläre Aufnahmen von Veranstaltungen oder Landschaften genutzt. Doch
    auch in anderen Bereichen wie dem Katastrophenmanagement oder der Such- und Rettungsarbeit
    haben sich Drohnen als nützliche Werkzeuge erwiesen. Jedoch sind Drohnen mittlerweile
    längst nicht mehr nur der Industrie und Fachleute vorbehalten, sondern können von jedem
    erworben werden.\\ 
    \\
    Eine besondere Art von Drohnen sind FPV-Drohnen (FPV - First Person View). Der Pilot
    steuert herkömmliche Drohnen meist aus der Ferne und bekommt das Bild aus einer mechanisch
    stabilisierten Kamera, welche sich am Rumpf der Drohne befindet, auf einen Bildschirm
    übertragen. Im Gegensatz dazu bieten FPV-Drohnen die Möglichkeit, das Flugerlebnis hautnah
    mitzuerleben. Der Pilot steuert die Drohne aus der Perspektive des Fluggeräts und bekommt
    so ein realistisches Fluggefühl vermittelt. Diese Art von Drohnen bieten eine immersive
    Flugerfahrung und ein hohes Maß an manueller Kontrolle erfordern jedoch viel Übung. \\
    \\
    Nicht nur der Flug, sondern auch der Bau einer FPV-Drohne stellt Piloten vor eine Reihe
    an Herausforderungen. Die Hardware muss robust und leistungsfähig genug sein, um den
    Flug zu ermöglichen, jedoch gleichzeitig klein und leicht genug, um die Agilität und
    Manövrierfähigkeit der Drohne nicht einzuschränken.\\
    \\
    Die folgende Arbeit beschäftigt sich mit dem Thema "FPV-Drohnen - Herausforderungen an Hard- und
    Software". Mit einer wissenschaftlichen Herangehensweise wird das immer weiter verbreitete
    Hobby und professionell eingesetzte Tool der FPV-Drohne analysiert. Speziell wird sich mit der
    Leitfrage beschäftigt, Inwieweit die einzelnen Komponenten der Drohne Einfluss auf die Videoqualität
    und das Flugverhalten nehmen und wie beides durch geschickte Wahl der Komponenten verbessert werden kann.\\
    \\
    Ziel der Arbeit ist es, einen umfassenden Überblick über die Herausforderungen an Hard- und Software bei
    der Nutzung von FPV-Drohnen zu geben. Die Funktion der wichtigsten Bauteile zu erläutern und darzustellen,
    wie durch die richtige Auswahl und Kombination von Komponenten die Leistung und Qualität der Drohne verbessert
    werden kann. Darüber hinaus wird aufgezeigt, welche Lösungen zur Bildstabilisierung im Bereich der Drohnen am
    verbreitetsten sind, welche Stabilisierung bei FPV-Drohnen zum Einsatz kommt und wie diese sinnvoll eingesetzt
    werden kann, um die Videoqualität zu verbessern. \\
\section[FPV-Drohnen]{FPV-Drohnen}
\subsection[Besonderheiten von FPV-Drohnen]{Besonderheiten von FPV-Drohnen}

Wie bereits in der Einleitung aufgegriffen, gibt es erhebliche Unterschiede, zwischen handelsüblichen Drohnen und FPV-Drohnen. In der folgenden Tabelle werden die wichtigsten Unterschiede gegenübergestellt. Im Anschluss werden diese genauer erklärt und verglichen. 
\\ \\ \\
\renewcommand{\arraystretch}{1.5}
\begin{tabular}{p{4cm}p{6cm}p{6cm}}
    \toprule
    \textbf{Merkmal} & \textbf{Handelsübliche - Drohnen} & \textbf{FPV - Drohnen} \\
    \midrule
    Steuerung           & X & X\\
    Sichtverhältnisse   & X & X\\
    Verwendungszweck    & X & X\\
    Flugmodi            & X & X\\
    Kamera              & X & X\\
    Leistung            & X & X\\
    \bottomrule
\end{tabular}







%\subsection[Einsatzmöglichkeiten von FPV-Drohnen]{Einsatzmöglichkeiten von FPV-Drohnen}

\section[Bauteile]{Bauteile}
    Sowohl FPV-Drohnen als auch normale Drohnen verwenden eine Vielzahl von Bauteilen, um Flüge zu ermöglichen und zu kontrollieren. Dazu gehören beispielsweise Propeller, Akkus, Steuerungen und Sensoren. All diese Bauteile haben, auch wenn sie in unterschiedlichen Drohnen zum Einsatz kommen die gleichen grundlegenden Funktionen, wie zum Beispiel die Bereitstellung von Antrieb und Strom, die Kontrolle der Flugbewegungen und die Navigation in der Umgebung. Jedoch gibt es enorme Unterschiede, die den Preis, die Leistung, den Anwendungsfall und viele weitere Faktoren betreffen.
    \\ \\
    Im weiteren Verlauf dieser Arbeit werden die Funktionen der Bauteile einer FPV-Drohne dargelegt, welche den größten Einfluss auf das Flugverhalten und die Videoqualität haben. Dazu gehören der Flight Controller (FC), der Rahmen und die Propeller. 

\subsection[FC - Flight Controller]{FC - Flight Controller}
    „Der Flight Controller ist das Herzstück eines Kopters und der Grund dafür, dass ein Kopter überhaupt fliegt“. Er ist eine elektronische Einheit, die dazu dient, die Flugbewegungen der Drohne zu kontrollieren und zu stabilisieren. Der Flight Controller ist mit Sensoren wie z.B. Gyroskopen und Beschleunigungsmessern ausgestattet, die ihm ermöglichen, die Position und Bewegung der Drohne zu messen und zu verarbeiten. Basierend auf diesen Messwerten und möglichen Befehlen des Nutzers, die über die Fernsteuerung übertragen werden, werden Steuersignale an die Motoren bzw. dem electronic speed controller der Drohne gesendet, um die Flugbewegungen auszuführen. Der FC kann auch mit zusätzlichen Komponenten wie GPS-Modulen oder einem Kompass ausgestattet werden, um zusätzliche Funktionen wie Navigation oder „Return to Home“ zu ermöglichen. Dies wird jedoch meist nur bei größeren FPV-Drohnen, die für den „Long Range“ Flug gedacht sind gemacht, um einen Zusätzlichen Schutz bei Signalabbruch gewährlisten zu können.
    \\ \\
    FPV-Flight Controller sind speziell für den Einsatz in FPV-Drohnen ausgelegt und haben einige Eigenschaften, die dies zeigen. Eines dieser Merkmale ist eine hohe Leistung und Berechnungsgeschwindigkeit. Dies führt zu einer geringen PID-Loop Dauer und nimmt so direkten Einfluss auf das Flugverhalten. Eine weitere Besonderheit, sind die frei zugänglichen und anpassbareren PID-Parameter (Proportional-Integral-Derivative), um die Flugsteuerung an die spezifischen Anforderungen und Vorlieben des Nutzers anpassen zu können.

\subsubsection[PID - Proportional-Integral-Derivative Loop]{PID - Proportional-Integral-Derivative Loop}
    Der PID-Loop (Proportional-Integral-Derivative Loop) wird in der Flugsteuerung von FPV-Drohnen verwendet, um die Flugbewegungen der Drohne zu stabilisieren und zu kontrollieren. Der PID-Loop besteht aus drei Hauptkomponenten:

    \begin{itemize}
        \item[1.] Proportional: Die proportionalen Steuerung bezieht sich auf die aktuelle Abweichung des Systems von seinem Sollwert. Je größer die Abweichung ist, desto stärker wird die Korrektur.
        \item[2.] Integral: Die Integralsteuerung bezieht sich auf die Summe aller Abweichungen des Systems von seinem Sollwert über einen bestimmten Zeitraum. Sie hilft dabei, kleinere Abweichungen auszugleichen und das System in seinem Sollzustand zu halten.
        \item[3.] Derivative: Die derivative Steuerung bezieht sich auf die Änderungsrate der Abweichung des Systems von seinem Sollwert. Sie hilft dabei, das System auf Änderungen in der Umgebung schneller zu reagieren und das System stabiler zu halten.
     \end{itemize}

     Der PID-Loop berechnet die Steuersignale für das System basierend auf den Werten der drei Komponenten und den Eingaben durch den Piloten. Anschließend werden die berechneten Steuerwerte an den ESC weitergegeben, um die Flugbewegungen der Drohne zu kontrollieren und zu stabilisieren. Die eingegebenen PID-Werte richten sich stark nach Gewicht und dem gewünschten Flugverhalten. Zusätzlich müssen allerdings auch Störgrößen wie Änderungen in der Umgebung, Verzögerungen in der Signalverarbeitung oder Einflüsse durch Vibrationen, welche Messwerte verfälschen können, beachtet werden.
     \\ \\
     Wenn die PID-Parameter falsch eingestellt werden, kann es im besten Fall zu einem instabilen oder trägen Flugverhalten der Drohne kommen. Im schlimmsten Fall besteht jedoch auch die Gefahr, dass die Drohne unkontrolliert in eine unerwartete Richtung fliegt und damit Personen oder Objekte in der Umgebung gefährdet. Dieses Phänomen nennt sich „Fly away“. Aus diesem Grund sind bei herkömmlichen Drohnen die PID-Werte für den Käufer nicht sichtbar und lassen sich auch nicht ändern. Dies ist bei FPV-Drohnen im Normallfall allerdings nicht möglich, da ein FC sowohl in einer schweren, langsame als auch in einer leichten und sehr agilen Drohne verbaut werden kann und so je nach Anwendungsfall vom Benutzer unterschiedlich parametriert werden muss.

\subsubsection[Filter]{Filter}
     Filter sind Hardware- und/oder Softwarekomponenten, welche sich auf dem FC befinden und ungewollte Vibrationen beziehungsweise Frequenzen, welche durch Sensoren aufgenommen werden minimieren. Diese Vibrationen entstehen durch aerodynamische Effekte und die Drehbewegung der Motoren und Propeller. Sie können die Werte des Gyroskops verfälschen und so zu einem unkontrollierbaren und unerwarteten Flugverhalten führen. Durch Filter kann dem entgegengewirkt werden, indem Frequenzen, welche nicht das Resultat einer Bewegung sind, sondern durch Vibrationen entstanden herausfiltert werden. Der Nachteil von Filtern liegt in der Verzögerung, die sie dem Signal hinzufügen. Dies sorgt dafür, dass eine Bewegung der Drohne den PID-Loop erst verzögert erreicht. Aus diesem Grund ist eine möglichst geringe, aber dennoch ausreichende Filterung erstrebenswert. Ein einfacher, jedoch sehr effektiver weg eine Filterung zu erreichen, ist das sogenannte „soft mounting“ des FCs. Hierbei wird der FC nicht direkt mit dem Rahmen verschraubt, sondern ist durch Gummis vom Rahmen entkoppelt. Experiment: mit/ohne gummi
     \\ \\
     
\subsection[Rahmen]{Rahmen}
    Der Rahmen, oft auch als „Frame“ bezeichnet bildet das Grundgerüst einer jeden FPV-Drohne. Er besteht aus einem leichten, aber robusten Material wie Kohlefaser und nimmt erheblichen Einfluss auf das Flugverhalten der Drohne. Die Größe des Rahmens beeinflusst nicht nur das Erscheinungsbild, sondern die gesamte Konstruktion der FPV-Drohne. Aus diesem Grund ist es unerlässlich, dass die Größe des Rahmens sorgfältig geplant und ausgewählt wird. Die Wahl des Rahmens bestimmt im weiteren Verlauf die Auswahl und Anordnung der weiteren Komponenten, wie Motoren, Propeller und Batterie, welche auf den Rahmen abgestimmt werden müssen, um ein optimales Flugverhalten zu gewährleisten.
    \\ \\
    Die Größe eines FPV-Drohnenrahmens wird in der Regel in Zoll (inch) angegeben und bezieht sich auf die maximale Größe der Propeller, welche an diesem verwendet werden können. Eine weitere, wichtige Kenngröße ist die Dicke des Rahmens. Ein dicker Rahmen sorgt für eine höhere Steifigkeit, was zu einem präziseren und stabileren Flug führt. Allerdings erhöht dieser auch das Gewicht und könnte so das Flugverhalten und die Manövrierfähigkeit beeinträchtigen. Ein dünner Rahmen ist leichter, bietet dadurch eine längere Akkulaufzeit und höhere Manövrierfähigkeit, kann allerdings auch anfälliger für Beschädigungen und durch die geringere Steifigkeit weniger stabil im Flug sein.
    \\ \\
    Zusätzlich haben die Dicke und Größe des Rahmens Einfluss auf die Weiterleitung von Vibrationen, welche durch die Motoren entstehen und sich in den Rahmen fortbewegen. Starke Vibrationen können zu Fehlern im PID-Loop führen, da Messwerte des Gyroskops und anderer Sensoren beeinträchtigt und verfälscht werden. Neben den Flugeigenschaften leidet auch die Videoqualität unter Vibrationen, welche die Kamera erreichen. Diese Vibrationen können durch einen dickeren Rahmen minimiert werden. Auch die Größe des Rahmens hat Einfluss auf die Vibrationen, so erfährt beispielsweise ein 7“ Rahmen deutlich mehr Vibrationen bei gleicher Dicke als ein 5“ Rahmen, da die Arme des 7“ Rahmens länger sind, was zu einer reduzierten Steifigkeit und somit mehr Vibrationen führt. Diesem Effekt kann durch eine passende Filterung entgegengewirkt werden.

\subsection[Propeller]{Propeller}
\section[Bildstabilisierung]{Bildstabilisierung}
    Es gibt verschiedene Arten der Bildstabilisierung. Im Zusammenhang mit Drohnen sind zwei Stabilisierungsarten am weitesten verbreitet. Die mechanische Bildstabilisierung und die digitale Bildstabilisierung.
    \\ \\
    Mechanische Bildstabilisierung bezieht sich auf die Verwendung von mechanischen Elementen, um die Bewegungen einer Kamera auszugleichen. Im Bereich der Drohnen wird dies durch die Verwendung eines Gimbals erreicht. Ein Gimbal ist ein dreiachsiges System, welches die Kamera stabil hält, indem es die Bewegungen auf jeder Achse bis zu einem gewissen Grad ausgleicht. Diese Art der Stabilisierung findet sich meist in herkömmlichen Drohnen.
    \\ \\
    Die digitale Bildstabilisierung kommt, aufgrund des geringeren Gewichts meist in FPV-Drohnen zum Einsatz. Jedoch gibt es unterschiedliche Methoden, um ein Video digital zu stabilisieren. Häufig werden Algorithmen zur Bewegungserkennung verwendet, welche die Bewegungen der Kamera verfolgen und die Bilder entsprechend korrigieren. Dies geschieht in Echtzeit und erfordert wenig technisches Vorwissen in der Benutzung. Eine weitere Methode zur digitalen Bildstabilisierung nutzen die Daten aus Bewegungssensoren, wie zum Beispiel Gyrosensoren, um die Bewegungen der Kamera zusätzliche zum eigentlichen Bild zu erfassen. Anschließend werden diese Daten miteinander verglichen und das Bild entsprechend der aufgezeichneten Bewegungen angepasst. Diese Arbeit wird jedoch nicht in Echtzeit, sondern meist in einer speziellen Software durchgeführt.
    \\ \\
    Obwohl die digitale Bildstabilisierung viele Vorteile bietet, besteht ein großer Nachteil im Vergleich zur mechanischen Stabilisierung. Der Qualitätsverlust. Da eine mechanische Stabilisierung die Kamera physikalisch bewegt und so beispielsweise immer horizontal halten kann, ist kein digitales Zoomen notwendig. Bei einer digitalen Stabilisierung muss zwangsläufig gezoomt werden, da der Rand des Videos durch die Stabilisierung beschnitten wird und sich so das Seitenverhältnis des Videos ändert. Um dem entgegenzuwirken, wird digital in das Bild gezoomt, bis das gewünschte Seitenverhältnis erreicht und das Bild ausgefüllt ist. Bei diesem Vorgang geht jedoch ein Teil der Auflösung verloren, was in folgender Grafik dargestellt ist.
    \\ \\
    Den Ausgangszustand und somit die Eingabe der Stabilisierung bildet ein um 90 Grad gedrehtes Bild einer nicht mechanisch stabilisierten Kamera, welches in der Abbildung in schwarz dargestellt ist. Das Bild hat aufgrund der 90 Grad Drehung das Format 9:16 und die Auflösung 1080x1920. Soll diese Bild nun horizontal im Seitenverhältnis 16:9 stabilisiert werden, kann eine, hier in blau dargestellte Maske, im gewünschten Seitenverhältnis über die Eingabe gelegt werden. Da die neue Maske jedoch 1920 und nicht 1080 Pixel in der Horizontalen besitzt, haben 840 Pixel pro Reihe keinen Wert und bilden somit schwarze Balken an den Seiten. Um dem entgegenzuwirken, wird die Maske auf die Breite der Eingabe skaliert. Da das Seitenverhältnis beibehalten werden soll, muss dies sowohl für die Horizontale als auch für die Vertikale gelten. Dieser Schritt ist in Grün dargestellt.



\newpage

\begin{figure}[h]
    \centering
    \def\svgwidth{\linewidth}
    \input{IMG/Grafik_StabilisierungV2.pdf_tex}
    \vspace{0.5cm}
    \caption{Test}
    \label{vektorgrafik}
\end{figure}

    Es lässt sich eindeutig erkenne, dass ein großer Teil der Auflösung fehlt, das Endresultat jedoch ein Bild im Format 16:9 ist. Dieses vorgehen lässt sich auf jedes Bild eines Videos anwenden, um ein Horizontal stabilisiertes Video zu erhalten. Bei einer geringeren Drehung der Eingabe, beispielsweise um 5 Grad, ist der Verlust aufgrund der Stabilisierung deutlich geringer. Mathematisch lässt sich der Unterschied zwischen Ein- und Ausgabe durch die folgende Formel beschreiben.

\begin{equation}
    \frac{A_x*A_y}{E_x*E_y}*100\%
\end{equation}


    Hierbei stehen $A_x$ und $A_y$ für die Länge und Breite der Ausgabe in Pixeln. $E_x$ und $E_y$ analog für die Länge und Breite der Eingabe. Das Ergebnis gibt, relativ zur Eingabe an, wie viel Prozent der ursprünglichen Pixel in der Ausgabe enthalten sind.








\end{document}