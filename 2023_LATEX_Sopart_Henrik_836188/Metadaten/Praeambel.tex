%%% Henrik Sopart, 836188, Latex-Projekt %%%%

\documentclass[a4paper, 11pt]{article} % Dokumentenklasse und Papierformat festlegen

\usepackage[left=2.5cm, right=2.5cm]{geometry} % Ränder festlegen
\usepackage{fancyhdr}   % Um Kopf- und Fußzeilen zu gestalten
\usepackage[utf8]{inputenc} % Encoding für Umlaute
\usepackage[T1]{fontenc}    % Font Encoding für korrekte Darstellung von Sonderzeichen
\usepackage[ngerman]{babel} % Sprachpaket für deutsche Sprache
\usepackage{lmodern}    % Schriftart
\usepackage{amsmath}    % Mathematische Symbole und Funktionen
\usepackage{amssymb}    % Mathematische Symbole und Funktionen
\usepackage{graphicx}   % Einbinden von Grafiken
\usepackage{caption}    % Bildunterschriften verbessern
\usepackage{subcaption} % Unter-Bildunterschriften
\usepackage{booktabs}   % Erstellen von "schöneren" Tabellen
\usepackage{xcolor}     % Farben
\usepackage{titlesec}   % Anpassen der Formatierung von Abschnittstiteln
\usepackage{tocloft}    % Anpassen des Inhaltsverzeichnisses
\usepackage{siunitx}    % Darstellung von Einheiten
\usepackage{csquotes }  % Für Zitate (nicht benutzt, aber ohne meckert vs)
\usepackage[backend=biber]{biblatex} % Literaturverzeichnis, biber wegen UTF-8
\usepackage[colorlinks=true, urlcolor=blue, linkcolor=black, citecolor=black]{hyperref} % Schöne "bunte" links und Metadaten

\setlength{\parindent}{0cm} % Verhindert das Einrücken
\captionsetup[figure]{justification=raggedright,singlelinecheck=false} % "justification=raggedright" -> Beschriftungen linksbündig ; singlelinecheck=false -> keine automatische Zentrierung
\captionsetup[table]{justification=raggedright,singlelinecheck=false} % "justification=raggedright" -> Beschriftungen linksbündig ; singlelinecheck=false -> keine automatische Zentrierung
\sisetup{locale = DE, per-mode = fraction, separate-uncertainty} % Darstellung von Einheiten, Trennung etc. auf deutsch

% Schrift, die Arial ähnlich sieht
\usepackage{helvet}
\renewcommand{\familydefault}{\sfdefault}

% Korrektur für Kopfzeile
\setlength{\headheight}{13.59999pt}
\addtolength{\topmargin}{-1.59999pt}

% Kopfzeile
\pagestyle{fancy}
\fancyhead[L]{}