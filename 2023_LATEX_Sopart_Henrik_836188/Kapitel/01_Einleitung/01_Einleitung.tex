%%% Henrik Sopart, 836188, Latex-Projekt %%%%

\section[Einleitung]{Einleitung}
    Drohnen, auch bekannt als unbemannte Luftfahrzeuge oder UAVs (Unmanned Aerial Vehicles),
    haben in den letzten Jahren enorm an Popularität gewonnen und Perspektiven erlaubt, welche
    vorher unmöglich schienen. Sie werden in der Landwirtschaft eingesetzt, um gezielt Düngemittel
    zu verteilen oder Schädlinge zu bekämpfen. Auch in der Vermessungstechnik haben sich Drohnen
    bewährt, um große Flächen schnell und präzise zu vermessen. In der Unterhaltungsindustrie werden
    sie oft für spektakuläre Aufnahmen von Veranstaltungen oder Landschaften genutzt. Doch
    auch in anderen Bereichen wie dem Katastrophenmanagement oder der Such- und Rettungsarbeit
    haben sich Drohnen als nützliche Werkzeuge erwiesen. Jedoch sind Drohnen mittlerweile
    längst nicht mehr nur der Industrie und Fachleute vorbehalten, sondern können von jedem
    erworben werden.\\ 
    \\
    Eine besondere Art von Drohnen sind FPV-Drohnen (FPV - First Person View). Der Pilot
    steuert herkömmliche Drohnen meist aus der Ferne und bekommt das Bild aus einer mechanisch
    stabilisierten Kamera, welche sich am Rumpf der Drohne befindet, auf einen Bildschirm
    übertragen. Im Gegensatz dazu bieten FPV-Drohnen die Möglichkeit, das Flugerlebnis hautnah
    mitzuerleben. Der Pilot steuert die Drohne aus der Perspektive des Fluggeräts und bekommt
    so ein realistisches Fluggefühl vermittelt. Diese Art von Drohnen bieten eine immersive
    Flugerfahrung und ein hohes Maß an manueller Kontrolle erfordern jedoch viel Übung. \\
    \\
    Nicht nur der Flug, sondern auch der Bau einer FPV-Drohne stellt Piloten vor eine Reihe
    an Herausforderungen. Die Hardware muss robust und leistungsfähig genug sein, um den
    Flug zu ermöglichen, jedoch gleichzeitig klein und leicht genug, um die Agilität und
    Manövrierfähigkeit der Drohne nicht einzuschränken.\\
    \\
    Die folgende Arbeit beschäftigt sich mit dem Thema "FPV-Drohnen - Herausforderungen an Hard- und
    Software". Mit einer wissenschaftlichen Herangehensweise wird das immer weiter verbreitete
    Hobby und professionell eingesetzte Tool der FPV-Drohne analysiert. Speziell wird sich mit der
    Leitfrage beschäftigt, Inwieweit die einzelnen Komponenten der Drohne Einfluss auf die Videoqualität
    und das Flugverhalten nehmen und wie beides durch geschickte Wahl der Komponenten verbessert werden kann.\\
    \\
    Ziel der Arbeit ist es, einen umfassenden Überblick über die Herausforderungen an Hard- und Software bei
    der Nutzung von FPV-Drohnen zu geben. Die Funktion der wichtigsten Bauteile zu erläutern und darzustellen,
    wie durch die richtige Auswahl und Kombination von Komponenten die Leistung und Qualität der Drohne verbessert
    werden kann. Darüber hinaus wird aufgezeigt, welche Lösungen zur Bildstabilisierung im Bereich der Drohnen am
    verbreitetsten sind, welche Stabilisierung bei FPV-Drohnen zum Einsatz kommt und wie diese sinnvoll eingesetzt
    werden kann, um die Videoqualität zu verbessern. \\