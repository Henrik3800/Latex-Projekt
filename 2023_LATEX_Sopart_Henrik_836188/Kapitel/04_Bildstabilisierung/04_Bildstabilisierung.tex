\section[Bildstabilisierung]{Bildstabilisierung}
    Es gibt verschiedene Arten der Bildstabilisierung. Im Zusammenhang mit Drohnen sind zwei Stabilisierungsarten am weitesten verbreitet. Die mechanische Bildstabilisierung und die digitale Bildstabilisierung.
    \\ \\
    Mechanische Bildstabilisierung bezieht sich auf die Verwendung von mechanischen Elementen, um die Bewegungen einer Kamera auszugleichen. Im Bereich der Drohnen wird dies durch die Verwendung eines Gimbals erreicht. Ein Gimbal ist ein dreiachsiges System, welches die Kamera stabil hält, indem es die Bewegungen auf jeder Achse bis zu einem gewissen Grad ausgleicht. Diese Art der Stabilisierung findet sich meist in herkömmlichen Drohnen.
    \\ \\
    Die digitale Bildstabilisierung kommt, aufgrund des geringeren Gewichts meist in FPV-Drohnen zum Einsatz. Jedoch gibt es unterschiedliche Methoden, um ein Video digital zu stabilisieren. Häufig werden Algorithmen zur Bewegungserkennung verwendet, welche die Bewegungen der Kamera verfolgen und die Bilder entsprechend korrigieren. Dies geschieht in Echtzeit und erfordert wenig technisches Vorwissen in der Benutzung. Eine weitere Methode zur digitalen Bildstabilisierung nutzen die Daten aus Bewegungssensoren, wie zum Beispiel Gyrosensoren, um die Bewegungen der Kamera zusätzliche zum eigentlichen Bild zu erfassen. Anschließend werden diese Daten miteinander verglichen und das Bild entsprechend der aufgezeichneten Bewegungen angepasst. Diese Arbeit wird jedoch nicht in Echtzeit, sondern meist in einer speziellen Software durchgeführt.
    \\ \\
    Obwohl die digitale Bildstabilisierung viele Vorteile bietet, besteht ein großer Nachteil im Vergleich zur mechanischen Stabilisierung. Der Qualitätsverlust. Da eine mechanische Stabilisierung die Kamera physikalisch bewegt und so beispielsweise immer horizontal halten kann, ist kein digitales Zoomen notwendig. Bei einer digitalen Stabilisierung muss zwangsläufig gezoomt werden, da der Rand des Videos durch die Stabilisierung beschnitten wird und sich so das Seitenverhältnis des Videos ändert. Um dem entgegenzuwirken, wird digital in das Bild gezoomt, bis das gewünschte Seitenverhältnis erreicht und das Bild ausgefüllt ist. Bei diesem Vorgang geht jedoch ein Teil der Auflösung verloren, was in folgender Grafik dargestellt ist.

\newpage

\begin{figure}[h]
    \centering
    \def\svgwidth{420pt}
    \input{IMG/Grafik_StabilisierungV2.pdf_tex}
    \vspace{0.5cm}
    \caption{Test}
    \label{vektorgrafik}
\end{figure}