%%% Henrik Sopart, 836188, Latex-Projekt %%%%

\section[Zusammenfassung und Ergebnis]{Zusammenfassung und Ergebnis}
Das Ziel dieser Arbeit war es, darlegen, inwieweit die einzelnen Komponenten einer FPV-Drohne
Einfluss auf die Videoqualität und das Flugverhalten nehmen und wie beides durch geschickte Wahl
der Komponenten verbessert werden kann. Durch einen experimentellen Aufbau wurde gezeigt, dass
bereits einfache Maßnahmen wie das „soft mounting“ des FCs und das korrekte Einstellen von Filtern
dazu beitragen, ungewollte Frequenzen, welche durch Vibrationen entstehen, zu minimieren, um so
das Flugverhalten zu verbessern. Zusätzlich wurde dargelegt, dass die Größe und Dicke des Rahmens
nicht nur auf das Flugverhalten, sondern auch auf die Videoqualität Einfluss nehmen und wie dieser
durch die Auswahl korrekter Komponenten minimiert werden kann. Es wurde außerdem aufgezeigt, dass
die Auswahl des Propellers einen entscheidenden Einfluss auf das Flugverhalten hat und durch
Berechnungen dargelegt, welchen Geschwindigkeiten und dementsprechend welchen Kräften die Propeller
ausgesetzt sind. Anschließend wurde anhand eines Beispiels gezeigt, dass die digitale
Bildstabilisierung ein enormes Potenzial, jedoch auch einen entscheiden Nachteil gegenüber einer
mechanischen Stabilisierung bietet. Neben dem Vorteil der Flexibilität im Schnitt bieten digitale
Bildstabilisierungen im Vergleich zu mechanischen enorme Vorteile im Bereich der Komplexität und
des Gewichts. Sie haben allerdings einen Nachteil gegenüber mechanisch stabilisierten Systemen,
der sich durch eine geringere Qualität des stabilisierten Videos auszeichnet. Da dieser Qualitätsverlust
jedoch durch korrekte Einstellung der Kamera minimiert oder eliminiert werden kann, bieten digitale
Lösungen zur Stabilisierung ein enormes Potenzial im Bereich der FPV-Drohnen. \\
\\
Durch diese Erkenntnis zeigt sich, dass Komponenten wie der FC, der Rahmen oder die Propeller zwar
einen enormen Einfluss auf das Flugverhalten und teilweise auch auf die Videoqualität haben, sich
jedoch nicht negativ gegenseitig beeinflussen. So werden beispielsweise durch einen dickeren Rahmen
oder korrekt balancierte Propeller Vibrationen minimiert, was sich sowohl in einem besseren Flugverhalten
als auch in einer besseren Videoqualität zeigt. Auf der anderen Seite kann es unter Umständen nötig sein,
beispielsweise einen Rahmen mit geringer Dicke zu verwenden, um die maximale Akkulaufzeit zu erhalten.
Dies würde sowohl die Videoqualität als auch das Flugverhalten negativ beeinflussen. Da moderne FCs
jedoch in der lange sind, einen großen Anteil dieser Störungen zu Filtern und Algorithmen mithilfe
von Gyroskopdaten die Möglichkeit bieten, ein Video digital zu stabilisieren, dass etwaige Vibrationen
und Flugfehler im finalen Schnitt nicht sichtbar sind, kann auch mit einer mechanisch nicht optimalen
FPV-Drohne ein gutes Ergebnis erzielt werden. Für das bestmögliche Ergebnis empfiehlt es sich jedoch,
eine mechanisch solide Grundlage zu verwenden, welche im Bestfall nur minimal durch Filter oder 
Bildstabilisierung unterstützt werden muss. \\
\\
Die, dieser Arbeit zugrundlegenden theoretischen Konzepte, Überlegungen und Schlussfolgerungen
wurden bereits in der Praxis erfolgreich angewendet. In dem nachfolgenden Video kann dies beobachtet
werden. Besonders hervorzuheben ist hierbei die Qualität der digitalen Bildstabilisierung. \\
\\
\href{https://www.youtube.com/watch?v=yI7aHrwKL-8}{https://www.youtube.com/watch?v=yI7aHrwKL-8}
