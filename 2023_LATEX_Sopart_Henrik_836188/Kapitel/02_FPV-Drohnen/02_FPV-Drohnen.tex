\section[FPV-Drohnen]{FPV-Drohnen}
\subsection[Besonderheiten von FPV-Drohnen]{Besonderheiten von FPV-Drohnen}

    FPV-Drohnen, auch bekannt als Racing- oder Freestyle-Drohnen besitzen erhebliche Unterschiede, im Vergleich zu handelsüblichen Drohnen. In der folgenden Tabelle werden die Unterschiede gegenübergestellt. Im Anschluss werden die wichtigsten genauer erklärt. 

    \vspace{0.5cm}
    \renewcommand{\arraystretch}{1.5}
    \begin{tabular}{p{3cm}p{5.86cm}p{5.86cm}}
    \toprule
    \textbf{Merkmal} & \textbf{Handelsübliche - Drohnen} & \textbf{FPV - Drohnen} \\
    \midrule
    Steuerung           & Viele Unterstützungen durch Assistenzsysteme wie beispielsweise GPS oder Hindernisserkennung und vorprogrammierte Flugmodi               & Keine Assistenzsysteme oder Flugmodi vorhanden \\
    Sichtverhältnisse   & Flug über Sichtlinie oder Bildschirm, welcher an der Fernsteuerung befestigt ist und das Videosignal empfängt                            & Flug mit einer FPV Brille, welche das Videosignal empfängt \\
    Verwendungszweck    & Fotografie, Videografie, Vermessungstechnik, Such- und Rettungsarbeiten, Landwirtschaft                                                  & Rennen, Wettbewerbe, Freestyle-Flüge, extrem dynamische Videografie \\
    Kamera              & Meist, mittels Gimbal stabilisierte Kamera, welche an der Unterseite der Drohne befestigt ist und sich nach oben und unten neigen lässt  & Meist, ein einem festen Winkel, vorne im Ramhen montiert (oft noch analog) \\
    Leistung            & Deutlich längere Akkulaufzeit, Deutlich Leistungsschwächere Motoren, gerningere Agilität                                                 & Kurze Akkulaufzeiten, Deutlich leistungsstärkere Motoren, extrem agil und Manövrierfähig \\
    Zeitaufwand         & Sehr gerning -> Drohne kaufen, Akkus laden, Anleitung lesen, Fliegen                                                                     & Extrem hoch -> Zusammenstellen der Komponenten, Bau der Drohne, Programmierung der Drohne, passendes Zubehör wie Fernsteuerung und Akkus finden, Üben im Simulator etc. \\
    \bottomrule
    \end{tabular}
    \vspace{0.5cm}

\subsection[Steuerung]{Steuerung}
    Die Steuerung einer FPV-Drohne unterscheidet sich im Wesentlichen in zwei Punkten von der einer herkömmlichen. Zum einen bietet die FPV-Drohne im Normalfall keinerlei Assistenzsysteme, zum anderen ist die Reaktion der Drohne auf, an der Fernsteuerung eingegebene Befehle eine andere. Sobald sich bei einer herkömmlichen Drohne beide Sticks der Fernsteuerung in der neutralen Position befinden (in der Mitte), bleibt die Drohne in der Luft stehen. Wird nun der rechte Stick bis auf das maximale nach vorne bewegt, neigt sich die Drohne in diese Richtung und beschleunigt. Da die Neigung durch diverse Assistenzsysteme begrenzt ist, besteht kein Risiko, dass die Drohne „vorne überkippt“. Wird nun die Fernsteuerung nicht weiter betätigt, bewegt sich der Stick wieder in die neutrale Position und die Drohne bleibt in der Luft stehen. Vereinfacht lässt sich darstellen, dass die Bewegung am Stick in einen Winkel für die Drohne resultiert. Die Leistung der Motoren wird automatisch angepasst, um den (in diesem Beispiel) den Vorwärtsflug auf gleichbleibender Höhe zu ermöglichen. Die Höhe wird mit dem linken Stick kontrolliert. Allerdings geschieht dies relativ zur benötigten Drehzahl, um die Drohne auf einer Stelle zu halten. Beispiel: Befindet sich der linke Stick in der neutralen Position (in der Mitte) drehen die Propeller so schnell, dass die Drohne weder sinkt noch steigt. Wird nun der Stick nach vorne bzw. hinten betätigt, erhöht bzw. verringert sich die Drehzahl um einige Prozent, damit die Drohne langsam steigen bzw. sinken kann. Bei einer herkömmlichen Drohne ist dementsprechend für keine Achse eine kontinuierliche Korrektur durch den Piloten notwendig. 
    \\ \\
    Im Vergleich hierzu würde eine FPV-Drohne, bei der der rechte Stick maximal nach vorne gedrückt wird, unkontrolliert nach „vorne kippen“ und sich um diese Achse drehen, bis es zum Absturz kommt. Zusätzlich gibt es bei FPV-Drohnen keine neutrale Stick Position, bei welcher die Drohne an der Stelle Stehen bleibt. Jeder Befehl der Fernsteuerung wird ausgeführt. Vereinfacht bedeutet dies, dass eine Bewegung am Stick nicht in einem Winkel, sondern in einer Drehgeschwindigkeit, um die gesteuerte Achse resultiert. Auch die Höhensteuerung funktioniert anders. Am einfachsten lässt sich dies mit einem Potentiometer vergleichen. Befindet sich der linke Stick ganz hinten, drehen die Propeller nicht. Wird diese nun nach vorne bewegt, erhöht sich die Drehzahl linear. Wird er nicht mehr betätigt, behält die Drohne diese Drehzahl bei. Da die Bewegung einer Achse nur durch direktes Eingreifen des Piloten zu stoppen ist, ist eine kontinuierliche Korrektur der Drohne zwingend notwendig.

\subsection[Komplexität und Risiko]{Komplexität und Risiko}
    Da die meisten herkömmlichen Drohnen für eine möglichst große Zielgruppe entwickelt wurden, sind keine technischen Fähigkeiten oder Vorkenntnisse erforderlich, um mit einer solchen Drohne Luftaufnahmen zu erstellen. Nachdem die Drohne gekauft wurde, müssen lediglich die Akkus geladen und ein Blick in die Anleitung geworfen werden, um mit dem ersten Flug zu starten. Durch die Unterstützung unterschiedlichster Sensoren, wie beispielsweise eine Hinderniserkennung, wird das Risiko für die Drohne minimiert. Zusätzlich erleichtern es Technologien wie Geofencing dem Piloten, im Ramen der rechtlichen Grenzen zu fliegen.
    \\ \\
    Da FPV-Drohnen immer populärer werden, haben sich in den letzten Jahren neue Möglichkeiten gefunden, in dieses Hobby einzusteigen. Seit 2021 bietet die Firma „DJI“ zwei FPV-Drohnen an. Diese müssen nicht selbst zusammengebaut werden und erfordern ähnlich wie herkömmliche Drohnen keinerlei technische Fähigkeiten oder Vorkenntnisse. Da diese Drohnen jedoch proprietäre Komponenten verwenden, auf welche der Käufer keinen Einfluss hat, wird auf diese im weiteren Verlauf dieser Arbeit nicht weiter eingegangen.
    \\ \\
    Bis vor einigen Jahren gab es nur die Möglichkeit sämtliche Komponenten wie Motoren, Controller, Kamera, Rahmen etc. einzeln zu kaufen und sich so seine Drohne selbst zu bauen. Dies erfordert ein hohes maß an technischem Wissen und technische Fähigkeiten. Allerdings erhält man die Möglichkeit, die Drohne individuell auf die Bedürfnisse des Piloten zuzuschneiden. Beispielsweise würde für eine Drohne, welche Aufnahmen in den Alpen macht die Akkulaufzeit eine Priorität sein und aus diesem Grunde schwächere, aber effizientere Motoren zum Einsatz kommen. Hingegen würde eine Drohne, welche Autos auf einer Rennstrecke verfolgen soll leistungsstärkere Motoren erhalten.


