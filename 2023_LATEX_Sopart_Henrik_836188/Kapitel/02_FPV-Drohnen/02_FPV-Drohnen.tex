\section[FPV-Drohnen]{FPV-Drohnen}
\subsection[Besonderheiten von FPV-Drohnen]{Besonderheiten von FPV-Drohnen}

FPV-Drohnen, auch bekannt als Racing- oder Freestyle-Drohnen besitzen erhebliche Unterschiede, im Vergleich zu handelsüblichen Drohnen. In der folgenden Tabelle werden die wichtigsten Unterschiede gegenübergestellt. Im Anschluss werden diese genauer erklärt und verglichen. 
\\ \\ \\
\renewcommand{\arraystretch}{1.5}
\begin{tabular}{p{3.2cm}p{6.4cm}p{6.4cm}}
    \toprule
    \textbf{Merkmal} & \textbf{Handelsübliche - Drohnen} & \textbf{FPV - Drohnen} \\
    \midrule
    Steuerung           & Viele Unterstützungen durch Assistenzsysteme wie beispielsweise GPS oder Hindernisserkennung und vorprogrammierte Flugmodi               & Keine Assistenzsysteme oder Flugmodi vorhanden \\
    Sichtverhältnisse   & Flug über Sichtlinie oder Bildschirm, welcher an der Fernsteuerung befestigt ist und das Videosignal empfängt                            & Flug mit einer FPV Brille, welche das Videosignal empfängt \\
    Verwendungszweck    & Fotografie, Videografie, Vermessungstechnik, Such- und Rettungsarbeiten, Landwirtschaft                                                  & Rennen, Wettbewerbe, Freestyle-Flüge, extrem dynamische Videografie \\
    Kamera              & Meist, mittels Gimbal stabilisierte Kamera, welche an der Unterseite der Drohne befestigt ist und sich nach oben und unten neigen lässt  & Meist, ein einem festen Winkel, vorne im Ramhen montiert (oft noch analog) \\
    Leistung            & Deutlich längere Akkulaufzeit, Deutlich Leistungsschwächere Motoren, gerningere Agilität                                                 & Kurze Akkulaufzeiten, Deutlich leistungsstärkere Motoren, extrem agil und Manövrierfähig \\
    \bottomrule
\end{tabular}







%\subsection[Einsatzmöglichkeiten von FPV-Drohnen]{Einsatzmöglichkeiten von FPV-Drohnen}
