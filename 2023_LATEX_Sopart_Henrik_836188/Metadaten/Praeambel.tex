\documentclass[a4paper, 11pt]{article} % Documentklasse und Papierformat festlegen

\usepackage[left=2.5cm, right=2.5cm]{geometry} % Ränder festlegen
\usepackage{fancyhdr} % Um Kopf- und Fußzeilen zu gestalten
\usepackage[utf8]{inputenc} % Encoding für Umlaute
\usepackage[T1]{fontenc} % Font Encoding für korrekte Darstellung von Sonderzeichen
\usepackage[ngerman]{babel} % Sprachpaket für deutsche Sprache
\usepackage{lmodern} % Schriftart
\usepackage{amsmath} % Mathematische Symbole und Funktionen
\usepackage{amssymb} % Mathematische Symbole und Funktionen
\usepackage{graphicx} % Einbinden von Grafiken
\usepackage{caption}
\usepackage{subcaption}
\usepackage{booktabs} % Erstellen von "schöneren" Tabellen
\usepackage{xcolor} % Farben
\usepackage{titlesec}
\usepackage{tocloft}
\usepackage{siunitx}
\usepackage[backend=biber]{biblatex}




\setlength{\parindent}{0cm} % Verhindert das Einrücken
\captionsetup[figure]{justification=raggedright,singlelinecheck=false} % "justification=raggedright" -> Beschriftungen linksbündig ; singlelinecheck=false -> keine automatische Zentrierung
\captionsetup[table]{justification=raggedright,singlelinecheck=false} % "justification=raggedright" -> Beschriftungen linksbündig ; singlelinecheck=false -> keine automatische Zentrierung
\sisetup{locale = DE, per-mode = fraction, separate-uncertainty}