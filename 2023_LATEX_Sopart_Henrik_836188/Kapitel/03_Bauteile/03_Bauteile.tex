\section[Bauteile]{Bauteile}
    Sowohl FPV-Drohnen als auch normale Drohnen verwenden eine Vielzahl von Bauteilen, um Flüge zu ermöglichen und zu kontrollieren. Dazu gehören beispielsweise Propeller, Akkus, Steuerungen und Sensoren. All diese Bauteile haben die gleichen grundlegenden Funktionen, wie zum Beispiel die Bereitstellung von Antrieb und Strom, die Kontrolle der Flugbewegungen und die Navigation in der Umgebung. Jedoch gibt es enorme Unterschiede, die den Preis, die Leistung, den Anwendungsfall und viele weitere Faktoren betreffen.
    \\ \\
    Im weiteren Verlauf dieser Arbeit werden die Funktionen der Bauteile einer FPV-Drohne dargelegt, welche den größten Einfluss auf das Flugverhalten und die Videoqualität haben. Dazu gehören beispielsweise der Flight Controller (FC) oder der Electronic Speed Controller (ESC). Zusätzlich wird dargelegt, wie unterschiedliche Arten eines Bauteiles Einfluss auf das Flugverhalten und die Videoqualität nehmen.

\subsection[FC - Flight Controller]{FC - Flight Controller}
    „Der Flight Controller ist das Herzstück eines Kopters und der Grund dafür, dass der Kopter überhaupt fliegt“. Er ist eine elektronische Einheit, die dazu dient, die Flugbewegungen der Drohne zu kontrollieren und zu stabilisieren. Der Flight Controller ist mit Sensoren wie z.B. Gyroskopen und Beschleunigungsmessern ausgestattet, die ihm ermöglichen, die Position und Bewegung der Drohne zu messen und zu verarbeiten. Basierend auf diesen Messwerten und möglicherweise auch auf Befehlen des Nutzers, die über die Fernsteuerung übertragen werden, werden dann Steuersignale an die Motoren bzw. dem ESC der Drohne gesendet, um die Flugbewegungen auszuführen. Er kann auch mit zusätzlichen Komponenten wie GPS-Modulen oder einem Kompass ausgestattet werden, um zusätzliche Funktionen wie Navigation oder „Return to Home“ zu ermöglichen. Dies wird jedoch meist nur bei größeren FPV-Drohnen, die für den „Long Range“ Flug gedacht sind gemacht, um einen Zusätzlichen Schutz bei Signalabbruch gewährlisten zu können.
    \\ \\
    FPV-Flight Controller sind speziell für den Einsatz in FPV-Drohnen ausgelegt und haben einige Eigenschaften, die dies zeigen. Eines dieser Merkmale ist eine hohe Leistung und Berechnungsgeschwindigkeit. Dies führt zu einer geringen PID-Loop Dauer und nimmt so direkten Einfluss auf das Flugverhalten. Eine weitere Besonderheit, sind die frei zugänglichen und anpassbareren PID-Parameter (Proportional-Integral-Derivative), um die Flugsteuerung an die spezifischen Anforderungen und Vorlieben des Nutzers anpassen zu können.

\subsubsection[PID - Proportional-Integral-Derivative Loop]{PID - Proportional-Integral-Derivative Loop}
    Der PID-Loop (Proportional-Integral-Derivative Loop) wird in der Flugsteuerung von FPV-Drohnen verwendet, um die Flugbewegungen der Drohne zu stabilisieren und zu kontrollieren. Der PID-Loop besteht aus drei Hauptkomponenten:

    \begin{itemize}
        \item[1.] Proportional: Die proportionalen Steuerung bezieht sich auf die aktuelle Abweichung des Systems von seinem Sollwert. Je größer die Abweichung ist, desto stärker wird die Korrektur.
        \item[2.]Integral: Die Integralsteuerung bezieht sich auf die Summe aller Abweichungen des Systems von seinem Sollwert über einen bestimmten Zeitraum. Sie hilft dabei, kleinere Abweichungen auszugleichen und das System in seinem Sollzustand zu halten.
        \item[3.] Derivative: Die derivative Steuerung bezieht sich auf die Änderungsrate der Abweichung des Systems von seinem Sollwert. Sie hilft dabei, das System auf Änderungen in der Umgebung schneller zu reagieren und das System stabiler zu halten.
     \end{itemize}

     Der PID-Loop berechnet die Steuersignale für das System basierend auf den Werten der drei Komponenten und den Eingaben durch den Piloten. Anschließend werden die berechneten Steuerwerte an den ESC weitergegeben, um die Flugbewegungen der Drohne zu kontrollieren und zu stabilisieren. Die eingegebenen PID-Werte richten sich stark nach Gewicht und dem gewünschten Flugverhalten. Zusätzlich müssen allerdings auch Störgrößen wie Änderungen in der Umgebung, Verzögerungen in der Signalverarbeitung oder Einflüsse durch Vibrationen, welche Messwerte verfälschen können, beachtet werden.
     \\ \\
     Wenn die PID-Parameter falsch eingestellt werden, wird genau das Gegenteil erreicht. Es kann im besten Fall zu einem instabilen oder trägen Flugverhalten der Drohne kommen. Im schlimmsten Fall besteht jedoch auch die Gefahr, dass die Drohne unkontrolliert in unerwartete Richtungen fliegt und damit Personen oder Objekte in der Umgebung gefährdet. Dieses Phänomen nennt sich „Fly away“ Aus diesem Grund sind bei herkömmlichen Drohnen die PID-Werte für den Käufer nicht sichtbar und lassen sich auch nicht ändern. Dies ist bei FPV-Drohnen im Normallfall allerdings nicht möglich, da ein FC sowohl in einer schweren, langsame als auch in einer leichten und sehr agilen Drohne verbaut werden kann.